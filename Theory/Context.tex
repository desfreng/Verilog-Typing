\documentclass{article}
\usepackage[a4paper, margin=1cm]{geometry}
\usepackage[english]{babel}
\usepackage[mono=false]{libertine}

\usepackage{hyperref}
\usepackage{csquotes}
\usepackage{ascii}
\usepackage{enumitem}

\usepackage{amssymb}
\usepackage{amsmath}
\usepackage{amsfonts}
\usepackage{mathtools}

\usepackage{ebproof}


\DeclareMathOperator{\dom}{dom}

\DeclareMathOperator{\Size}{Size}
\DeclareMathOperator{\Add}{Add}
\DeclareMathOperator{\Id}{Id}
\DeclareMathOperator{\Mul}{Mul}

\allowdisplaybreaks{}

\newcommand{\mytilde}{\raisebox{-.7ex}{\asciitilde{}}}

\renewcommand{\epsilon}{\varepsilon}

\newcommand{\binOp}{\ensuremath{\left\{
      \texttt{+}, \texttt{-}, \texttt{*}, \texttt{/}, \texttt{\%}, \texttt{\&},
      \texttt{|}, \texttt{\^{}}, \texttt{\^{}}\mytilde, \mytilde\texttt{\^{}}
      \right\}}}
\newcommand{\unOp}{\ensuremath{\left\{
      \texttt{+}, \texttt{-}, \texttt{++}, \texttt{--}, \mytilde
    \right\}}}
\newcommand{\cast}{\ensuremath{\left\{\texttt{\$signed}, \texttt{\$unsigned}\right\}}}
\newcommand{\comp}{\ensuremath{\left\{
      \texttt{===}, \texttt{!==}, \texttt{==?}, \texttt{!=?}, \texttt{==},
      \texttt{!=}, \texttt{>}, \texttt{>=}, \texttt{<}, \texttt{<=}
    \right\}}}
\newcommand{\logic}{\ensuremath{\left\{\texttt{\&\&}, \texttt{||}, \texttt{->}, \texttt{<->}\right\}}}
\newcommand{\red}{\ensuremath{\left\{
      \texttt{\&}, \mytilde\texttt{\&}, \texttt{|}, \mytilde\texttt{|}, \texttt{\^{}},
      \mytilde\texttt{\^{}}, \texttt{\^{}}\mytilde, \texttt{!}
    \right\}}}
\newcommand{\shift}{\ensuremath{\left\{
      \texttt{>>}, \texttt{<}\texttt{<}, \texttt{**}, \texttt{>>>},
      \texttt{<}\texttt{<}\texttt{<}
    \right\}}}
\newcommand{\assignBinOp}{\ensuremath{\left\{
      \texttt{=}, \texttt{+=}, \texttt{-=}, \texttt{*=}, \texttt{/=},
      \texttt{\%=}, \texttt{\&=}, \texttt{|=}, \texttt{\^{}=}
    \right\}}}
\newcommand{\assignShift}{\ensuremath{\left\{
      \texttt{<}\texttt{<=}, \texttt{>>=},
      \texttt{<}\texttt{<}\texttt{<=}, \texttt{>>>=}
    \right\}}}

\pagestyle{empty}

\begin{document}

\section*{Verilog Size Checking}


\subsection*{Expressions}

\begin{itemize}
    \item $\mathcal{A}$ is the set of atoms in an expression. In the AST, they correspond to leaves. In System-Verilog, they refer to what the standard calls an ``operand'' (variables, integers, function calls, slices of a variable, etc.).
    \item $\mathcal{E}$ is the set of System-Verilog expressions. We have $\mathcal{A} \subset \mathcal{E}$. An expression either contains an expression or is an atom.
\end{itemize}

\subsection*{Contexts}
\begin{itemize}
    \item Set of contexts identifiers: $\mathcal{C} \simeq \mathbb{N}$.
    \item A context $q \subseteq \mathcal{Q}$ with:
          \begin{equation*}
              \mathcal{Q} \coloneq
              \underbrace{\left\{\Size s \mid s \in \mathbb{N}\right\}}_{\text{Atomic size}}
              \sqcup
              \underbrace{\left\{\Id c \mid c \in \mathcal{C}\right\}}_{\text{Identity dependencies}}
              \sqcup
              \underbrace{\left\{\Add~(c_1, c_2) \mid c_1 \in \mathcal{C}, c_2 \in \mathcal{C}\right\}}_{\text{Additive dependencies}}
              \sqcup
              \underbrace{\left\{\Mul~(s, c) \mid c \in \mathcal{C}, s \in \mathbb{N}\right\}}_{\text{Multiplicative dependencies}}
          \end{equation*}
          A context represent all the information needed to compute the size of associated expressions.

    \item $\Pi: \mathcal{C} \rightharpoonup  2^{\mathcal{Q}}$ a partial mapping from context identifiers to
          the smallest possible context set. This means that:
          \begin{equation*}
              \Pi(c) \subset \Pi'(c) \implies \Pi(c) = \Pi'(c)
          \end{equation*}
\end{itemize}

\subsection*{Rules}
The statement $\Pi \vdash e : c$ means:
\begin{center}
    In the context environment $\Pi$, all the information needed to compute the size of the expression $e$ is in the context-set identified by $c$ in $\Pi$.
\end{center}


We use the following notations:
\begin{itemize}
    \item $\Gamma$ compute the size of an atom.
    \item $\Phi$ compute the size of a lvalue.
\end{itemize}


\subsubsection*{Base case}

\begin{equation*}
    \begin{prooftree}
        \hypo{s = \Gamma(e)}
        \hypo{\Pi = \left\{c \mapsto q\right\}}
        \hypo{\Size s \in q}
        \hypo{e \in \mathcal{A}}
        \infer4{\Pi\vdash e: c}
    \end{prooftree}
\end{equation*}

\subsubsection*{Operators}

\begin{itemize}[leftmargin=*]
    \setlength{\itemsep}{2em}

    \item $\oplus \in \binOp$:

          \begin{equation*}
              \begin{prooftree}
                  \hypo{\Pi \vdash a: c}
                  \hypo{\Pi \vdash b: c}
                  \infer2{\Pi \vdash a \oplus b: c}
              \end{prooftree}
          \end{equation*}

    \item $\oplus \in \unOp$:

          \begin{equation*}
              \begin{prooftree}
                  \hypo{\Pi \vdash e: c}
                  \infer1{\Pi \vdash \oplus e: c}
              \end{prooftree}
          \end{equation*}

    \item $\oplus \in \cast$:

          \begin{equation*}
              \begin{prooftree}
                  \hypo{\Pi' \vdash e: c'}
                  \hypo{c \notin \dom \Pi'}
                  \hypo{\Pi = \Pi' \sqcup \left\{c \mapsto q\right\}}
                  \hypo{\Id c' \in q}
                  \infer4{\Pi \vdash \oplus \left(e\right): c}
              \end{prooftree}
          \end{equation*}

    \item $\oplus \in \comp$:

          \begin{equation*}
              \begin{prooftree}
                  \hypo{\Pi' \vdash a: c'}
                  \hypo{\Pi' \vdash b: c'}
                  \hypo{c \notin \dom \Pi'}
                  \hypo{\Pi = \Pi' \sqcup \left\{c \mapsto q\right\}}
                  \hypo{\Size 1 \in q}
                  \infer5{\Pi \vdash a \oplus b: c}
              \end{prooftree}
          \end{equation*}

    \item $\oplus \in \logic$:

          \begin{equation*}
              \begin{prooftree}
                  \hypo{\Pi_1 \vdash a: c_1}
                  \hypo{\Pi_2 \vdash b: c_2}
                  \hypo{\dom \Pi_1 \cap \dom \Pi_2 = \emptyset}
                  \hypo{c \notin \dom \Pi_1 \sqcup \dom \Pi_2}
                  \hypo{\Pi = \Pi_1 \sqcup \Pi_2 \sqcup \left\{c \mapsto q\right\}}
                  \hypo{\Size 1 \in q}
                  \infer6{\Pi \vdash a \oplus b: c}
              \end{prooftree}
          \end{equation*}

    \item $\oplus \in \red$:

          \begin{equation*}
              \begin{prooftree}
                  \hypo{\Pi' \vdash a: c'}
                  \hypo{c \notin \dom \Pi'}
                  \hypo{\Pi = \Pi' \sqcup \left\{c \mapsto q\right\}}
                  \hypo{\Size 1 \in q}
                  \infer4{\Pi \vdash \oplus a: c}
              \end{prooftree}
          \end{equation*}

    \item $\oplus \in \shift$:

          \begin{equation*}
              \begin{prooftree}
                  \hypo{\Pi_1 \vdash a: c}
                  \hypo{\Pi_2 \vdash b: c'}
                  \hypo{\Pi = \Pi_1 \sqcup \Pi_2}
                  \hypo{\dom \Pi_1 \cap \dom \Pi_2 = \emptyset}
                  \infer4{\Pi \vdash a \oplus b: c}
              \end{prooftree}
          \end{equation*}

    \item $\oplus \in \assignBinOp$:

          \begin{equation*}
              \begin{prooftree}
                  \hypo{s = \Phi(l)}
                  \hypo{\Pi' \vdash b: c'}
                  \hypo{\Size s \in \Pi'(c')}
                  \hypo{c \notin \dom \Pi'}
                  \hypo{\Pi = \Pi' \sqcup \left\{c \mapsto q\right\}}
                  \hypo{\Size s \in q}
                  \infer6{\Pi \vdash l \oplus b: c}
              \end{prooftree}
          \end{equation*}

    \item $\oplus \in \assignShift$:

          \begin{equation*}
              \begin{prooftree}
                  \hypo{s = \Phi(l)}
                  \hypo{\Pi' \vdash b: c'}
                  \hypo{c \notin \dom \Pi'}
                  \hypo{\Pi = \Pi' \sqcup \left\{c \mapsto q\right\}}
                  \hypo{\Size s \in q}
                  \infer5{\Pi \vdash l \oplus b: c}
              \end{prooftree}
          \end{equation*}


    \item If expression:

          \begin{equation*}
              \begin{prooftree}
                  \hypo{\Pi_1 \vdash e: c'}
                  \hypo{\Pi_2 \vdash a: c}
                  \hypo{\Pi_2 \vdash b: c}
                  \hypo{\Pi = \Pi_1 \sqcup \Pi_2}
                  \hypo{\dom \Pi_1 \cap \dom \Pi_2 = \emptyset}
                  \infer5{\Pi \vdash e \texttt{?} a \texttt{:} b: c}
              \end{prooftree}
          \end{equation*}


    \item Concatenation:

          \begin{equation*}
              \begin{prooftree}
                  \hypo{\Pi_1 \vdash e_1: c_1}
                  \hypo{\Pi_2 \vdash e_2: c_2}
                  \hypo{\dom \Pi_1 \cap \dom \Pi_2 = \emptyset}
                  \hypo{c \notin \dom \Pi_1 \sqcup \dom \Pi_2}
                  \hypo{\Pi = \Pi_1 \sqcup \Pi_2 \sqcup \left\{c \mapsto q\right\}}
                  \hypo{\Add~(c_1, c_2) \in q}
                  \infer6{\Pi \vdash \texttt{\{}e_1, e_2 \texttt{\}}: c}
              \end{prooftree}
          \end{equation*}


    \item Replication:

          \begin{equation*}
              \begin{prooftree}
                  \hypo{i \in \mathbb{N}}
                  \hypo{\Pi' \vdash e: c'}
                  \hypo{c \notin \dom \Pi'}
                  \hypo{\Pi = \Pi' \sqcup \left\{c \mapsto q\right\}}
                  \hypo{\Mul~(i, c') \in q}
                  \infer5{\Pi \vdash \texttt{\{}i~e\texttt{\}}: c}
              \end{prooftree}
          \end{equation*}
\end{itemize}

\end{document}
