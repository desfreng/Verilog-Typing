\documentclass{article}
\usepackage[a4paper, margin=1cm]{geometry}
\usepackage[english]{babel}
\usepackage[mono=false]{libertine}

\usepackage{hyperref}
\usepackage{csquotes}
\usepackage{ascii}

\usepackage{amssymb}
\usepackage{amsmath}
\usepackage{amsfonts}
\usepackage{mathtools}

\usepackage{ebproof}

\newcommand{\mytilde}{\raisebox{-.7ex}{\asciitilde{}}}
\newcommand{\A}{\mathrm{A}}
\newcommand{\B}{\mathrm{B}}
\newcommand{\E}{\mathrm{E}}

\renewcommand{\epsilon}{\varepsilon}

\newcommand{\binOp}{\ensuremath{\left\{
      \texttt{+}, \texttt{-}, \texttt{*}, \texttt{/}, \texttt{\%}, \texttt{\&},
      \texttt{|}, \texttt{\^{}}, \texttt{\^{}}\mytilde, \mytilde\texttt{\^{}}
      \right\}}}
\newcommand{\preUnOp}{\ensuremath{\left\{
      \texttt{+}, \texttt{-}, \texttt{++}, \texttt{--}, \mytilde
    \right\}}}
\newcommand{\postUnOp}{\ensuremath{\left\{\texttt{++}, \texttt{--}\right\}}}
\newcommand{\cast}{\ensuremath{\left\{
      \texttt{\$signed}, \texttt{\$unsigned}, \texttt{signed'}, \texttt{unsigned'}
    \right\}}}
\newcommand{\comp}{\ensuremath{\left\{
      \texttt{===}, \texttt{!==}, \texttt{==?}, \texttt{!=?}, \texttt{==},
      \texttt{!=}, \texttt{>}, \texttt{>=}, \texttt{<}, \texttt{<=}
    \right\}}}
\newcommand{\logic}{\ensuremath{\left\{
      \texttt{\&\&}, \texttt{||}, \texttt{->}, \texttt{<->}
    \right\}}}
\newcommand{\red}{\ensuremath{\left\{
      \texttt{\&}, \mytilde\texttt{\&}, \texttt{|}, \mytilde\texttt{|}, \texttt{\^{}},
      \mytilde\texttt{\^{}}, \texttt{\^{}}\mytilde, \texttt{!}
    \right\}}}
\newcommand{\shift}{\ensuremath{\left\{
      \texttt{>>}, \texttt{<}\texttt{<}, \texttt{**}, \texttt{>>>},
      \texttt{<}\texttt{<}\texttt{<}
    \right\}}}
\newcommand{\assignBinOp}{\ensuremath{\left\{
      \texttt{=}, \texttt{+=}, \texttt{-=}, \texttt{*=}, \texttt{/=},
      \texttt{\%=}, \texttt{\&=}, \texttt{|=}, \texttt{\^{}=}
    \right\}}}
\newcommand{\assignShift}{\ensuremath{\left\{
      \texttt{<}\texttt{<=}, \texttt{>>=},
      \texttt{<}\texttt{<}\texttt{<=}, \texttt{>>>=}
    \right\}}}

\pagestyle{empty}

\begin{document}

\section*{Verilog Size Checking}

\subsection*{Equivalence-Based Typing Rules}

The statement $\Gamma \mid n \vdash e: \tau, X$ mean:
\begin{center}
  In the variable context $\Gamma$ and the size context $n$, the size of the expression $e$ is the value of the variable $\tau$,\\
  with $X$ the size-tagged set of equivalence classes of size
  variables.
\end{center}

So we have:
\begin{itemize}
  \item $\Gamma: \mathcal{O} \to \mathbb{N}$ a mapping from operand to
        their size, with $\mathcal{O}$ the set of operands,
  \item $n \in \mathbb{N} \cup \left\{\ast\right\}$ a size context (with
        $\ast$ for the empty one),
  \item $e \in \mathcal{E}$, with $\mathcal{E}$ the set of expressions,
  \item $\tau \in \mathcal{V}$, with $\mathcal{V}$ the set of size
        variable,
  \item $X = (S, f)$ with $S \in 2^{\mathcal{V}}$ the set of equivalence
        classes and $f: S \to \mathbb{N}$ a size valuation for each
        equivalence class.
\end{itemize}

We use the following notations:
\begin{itemize}
  \item \begin{align*} {}\bowtie{} & : \left\{ \def\arraystretch{1.2}
               \begin{array}{ccc}
                         \mathbb{N} \times \left(\mathbb{N} \cup \left\{\ast\right\}\right) & \to     & \mathbb{N} \\
                         (m, \ast)                                                          & \mapsto & m          \\
                         (m, n)                                                             & \mapsto & \max(m, n)
                       \end{array}
               \right.
        \end{align*}
  \item $\Delta_X : \mathcal{V} \to \mathbb{N}$ the function that given
        a size variable $x$ returns its size in the environment $X=(S,
          f)$. We have $\Delta_X(x) = f\left({[x]}_S\right)$ with ${[x]}_S$
        the equivalence class of $x$ in $S$.

  \item $\left\{\upsilon \coloneq s\right\}$: the declaration of a
        \emph{fresh} size variable $\upsilon$.
        \begin{align*}
          \left\{\upsilon \coloneq s\right\}
            & = (S, f)                     \\
          S & = \bigl\{\{\upsilon\}\bigr\} \\
          f & = \left\{
          \def\arraystretch{1.2}
          \begin{array}{ccc}
            S            & \to     & \mathbb{N} \\
            \{\upsilon\} & \mapsto & s
          \end{array}
          \right.
        \end{align*}

  \item $X/\alpha \sim \beta$ the operation that combines two
        classes. For the newly created class, the valuation function gives
        the maximum of the previous classes:
        \begin{align*}
          (S, f)/\alpha \sim \beta
             & = (S', f')                                           \\
          S' & =\bigl\{ {[\alpha]}_{S} \sqcup {[\beta]}_{S} \bigr\}
          \sqcup
          S\setminus \bigl\{{[\alpha]}_{S}, {[\beta]}_{S}\bigr\}
          \\
          f' & = \left\{
          \def\arraystretch{1.2}
          \begin{array}{cccl}
            S' & \to
               & \mathbb{N}
            \\
            c  & \mapsto
               & f(c)
               & \text{if } c \in S\setminus \bigl\{{[\alpha]}_{S}, {[\beta]}_{S}\bigr\}
            \\
            c  & \mapsto
               & \max\left(f\left({[\alpha]}_{S}\right), f\left({[\beta]}_{S}\right)\right)
               & \text{if } c = {[\alpha]}_{S} \text{ or } c = {[\beta]}_{S}
          \end{array}
          \right.
        \end{align*}

\end{itemize}

\subsubsection*{Base case}

\begin{equation*}
  \begin{prooftree}
    \hypo{\Gamma(x) = m} \infer1{\Gamma \mid n \vdash x: \upsilon,
      \left\{\upsilon \coloneq m \bowtie n\right\}}
  \end{prooftree}
\end{equation*}

\subsubsection*{Operators}

\begin{equation*}
  \def\arraystretch{4}
  \begin{array}{cc}
    \oplus \in \binOp
     &
    \begin{prooftree}
      \hypo{\Gamma \mid n \vdash a: \alpha,~\A} \hypo{\Gamma \mid n
        \vdash b: \beta,~\B} \infer2{\Gamma \mid n \vdash a \oplus
        b: \alpha,~\A \sqcup \B/\alpha \sim \beta}
    \end{prooftree}
    \\
    \oplus \in \preUnOp
     &
    \begin{prooftree}
      \hypo{\Gamma \mid n \vdash e: \epsilon,~\E}
      \infer1{\Gamma \mid n \vdash \oplus e: \epsilon,~\E}
    \end{prooftree}
    \\
    \oplus \in \postUnOp
     &
    \begin{prooftree}
      \hypo{\Gamma \mid n \vdash e: \epsilon,~\E}
      \infer1{\Gamma \mid n \vdash e \oplus : \epsilon,~\E}
    \end{prooftree}
    \\
    \oplus \in \cast
     &
    \begin{prooftree}
      \hypo{\Gamma \mid \ast \vdash e: \epsilon,~\E}
      \infer1{\Gamma \mid n \vdash \oplus \left(e\right) :
        \upsilon,~\left\{\upsilon \coloneq
        \Delta_{\E}(\epsilon) \bowtie n\right\}}
    \end{prooftree}
    \\
    \oplus \in \comp
     &
    \begin{prooftree}
      \hypo{\Gamma \mid \ast \vdash a: \alpha,~\A}
      \hypo{\Gamma \mid \ast \vdash b: \beta,~\B}
      \hypo{\A \sqcup \B/\alpha \sim \beta}
      \infer3{\Gamma \mid n \vdash a \oplus b: \upsilon,~\left\{\upsilon \coloneq 1 \bowtie n\right\}}
    \end{prooftree}
    \\
    \oplus \in \logic
     &
    \begin{prooftree}
      \hypo{\Gamma \mid \ast \vdash a: \alpha,~\A} \hypo{\Gamma \mid
        \ast \vdash b: \beta,~\B} \infer2{\Gamma \mid n \vdash a
        \oplus b: \upsilon,~\left\{\upsilon \coloneq 1\bowtie
        n\right\}}
    \end{prooftree}
    \\
    \oplus \in \red
     &
    \begin{prooftree}
      \hypo{\Gamma \mid \ast \vdash e: \epsilon,~\E}
      \infer1{\Gamma \mid n \vdash \oplus e:
        \epsilon,~\left\{\upsilon \coloneq 1 \bowtie
        n\right\}}
    \end{prooftree}
    \\
    \oplus \in \shift
     &
    \begin{prooftree}
      \hypo{\Gamma \mid n \vdash a: \alpha,~\A} \hypo{\Gamma
        \mid \ast \vdash b: \beta,~\B} \infer2{\Gamma \mid n
        \vdash a \oplus b: \alpha,~\A}
    \end{prooftree}
    \\
    \oplus \in \assignBinOp
     &
    \begin{prooftree}
      \hypo{\Gamma \mid \ast \vdash a: \alpha,~\A} \hypo{\Gamma \mid
        \Delta_{\A}(\alpha) \vdash b: \beta,~\B} \infer2{\Gamma \mid
        n \vdash a \oplus b: \upsilon,~\left\{\upsilon \coloneqq
        \Delta_{\A}(\alpha) \bowtie n\right\}}
    \end{prooftree}
    \\
    \oplus \in \assignShift
     &
    \begin{prooftree}
      \hypo{\Gamma \mid \ast \vdash a: \alpha,~\A}
      \hypo{\Gamma \mid c \ast \vdash b: \beta,~\B}
      \infer2{\Gamma \mid n \vdash a \oplus b:
        \upsilon,~\left\{\upsilon \coloneqq
        \Delta_{\A}(\alpha) \bowtie n\right\}}
    \end{prooftree}
  \end{array}
\end{equation*}

\begin{spreadlines}{2em}
  \begin{gather*}
    \begin{prooftree}
      \hypo{\Gamma \mid \ast \vdash e: \epsilon,~\E} \hypo{\Gamma \mid
        n \vdash a: \alpha,~\A} \hypo{\Gamma \mid n \vdash b:
        \beta,~\B} \infer3{\Gamma \mid n \vdash e \texttt{?} a
        \texttt{:} b: \alpha,~\A \sqcup \B / \alpha \sim \beta}
    \end{prooftree}
    \\
    \begin{prooftree}
      \hypo{\Gamma \mid \ast \vdash e_1: \epsilon_1,~\E_1}
      \hypo{\dots} \hypo{\Gamma \mid \ast \vdash e_n:
        \epsilon_n,~\E_n} \infer3{\Gamma \mid n \vdash \texttt{\{}e_1,
        \texttt{\dots}, e_n \texttt{\}} : \upsilon,~\left\{\upsilon
        \coloneq \left(\Delta_{\E_1}(\epsilon_1) + \cdots +
        \Delta_{\E_n}(\epsilon_n)\right)\bowtie n\right\}}
    \end{prooftree}
    \\
    \begin{prooftree}
      \hypo{i \in \mathbb{N}} \hypo{\Gamma \mid \ast \vdash e_1:
        \epsilon_1,~\E_1} \hypo{\dots} \hypo{\Gamma \mid \ast \vdash
        e_k: \epsilon_k,~\E_k} \infer4{\Gamma \mid n \vdash
        \texttt{\{}i\texttt{\{}e_1, \texttt{\dots}, e_k
        \texttt{\}}\texttt{\}} : \upsilon,~\left\{\upsilon \coloneq
        i\times\left(\Delta_{\E_1}(\epsilon_1) + \cdots +
        \Delta_{\E_k}(\epsilon_k)\right) \bowtie n\right\}}
    \end{prooftree}
  \end{gather*}
\end{spreadlines}

\end{document}
