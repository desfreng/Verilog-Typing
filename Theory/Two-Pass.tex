\documentclass{article}
\usepackage[a4paper, margin=1cm]{geometry}
\usepackage[english]{babel}
\usepackage[mono=false]{libertine}

\usepackage{hyperref}
\usepackage{csquotes}
\usepackage{ascii}
\usepackage{enumitem}

\usepackage{amssymb}
\usepackage{amsmath}
\usepackage{amsfonts}
\usepackage{mathtools}

\usepackage{ebproof}

\newcommand{\mytilde}{\raisebox{-.7ex}{\asciitilde{}}}

\renewcommand{\epsilon}{\varepsilon}

\newcommand{\binOp}{\ensuremath{\left\{
      \texttt{+}, \texttt{-}, \texttt{*}, \texttt{/}, \texttt{\%}, \texttt{\&},
      \texttt{|}, \texttt{\^{}}, \texttt{\^{}}\mytilde, \mytilde\texttt{\^{}}
      \right\}}}
\newcommand{\unOp}{\ensuremath{\left\{
      \texttt{+}, \texttt{-}, \texttt{++}, \texttt{--}, \mytilde
    \right\}}}
\newcommand{\cast}{\ensuremath{\left\{\texttt{\$signed}, \texttt{\$unsigned}\right\}}}
\newcommand{\comp}{\ensuremath{\left\{
      \texttt{===}, \texttt{!==}, \texttt{==?}, \texttt{!=?}, \texttt{==},
      \texttt{!=}, \texttt{>}, \texttt{>=}, \texttt{<}, \texttt{<=}
    \right\}}}
\newcommand{\logic}{\ensuremath{\left\{\texttt{\&\&}, \texttt{||}, \texttt{->}, \texttt{<->}\right\}}}
\newcommand{\red}{\ensuremath{\left\{
      \texttt{\&}, \mytilde\texttt{\&}, \texttt{|}, \mytilde\texttt{|}, \texttt{\^{}},
      \mytilde\texttt{\^{}}, \texttt{\^{}}\mytilde, \texttt{!}
    \right\}}}
\newcommand{\shift}{\ensuremath{\left\{
      \texttt{>>}, \texttt{<}\texttt{<}, \texttt{**}, \texttt{>>>},
      \texttt{<}\texttt{<}\texttt{<}
    \right\}}}
\newcommand{\assignBinOp}{\ensuremath{\left\{
      \texttt{=}, \texttt{+=}, \texttt{-=}, \texttt{*=}, \texttt{/=},
      \texttt{\%=}, \texttt{\&=}, \texttt{|=}, \texttt{\^{}=}
    \right\}}}
\newcommand{\assignShift}{\ensuremath{\left\{
      \texttt{<}\texttt{<=}, \texttt{>>=},
      \texttt{<}\texttt{<}\texttt{<=}, \texttt{>>>=}
    \right\}}}

\newcommand{\type}[3]{\vdash#2:#1[#3]}

\newcommand{\vDown}{\vdash\downarrow}


\pagestyle{empty}

\begin{document}

\section*{Two Pass Typing Rules}

\subsection*{Rules}
% The statement $\Pi \ e : c$ means:
% \begin{center}
%     In the context environment $\Pi$, all the information needed to compute the size of the expression $e$ is in the context-set identified by $c$ in $\Pi$.
% \end{center}


We use the following notations:
\begin{itemize}
    \item $\Gamma$ compute the size of an atom.
    \item $\Phi$ compute the size of a lvalue.
\end{itemize}


\subsubsection*{Base cases}

\begin{equation*}
    \begin{prooftree}
        \hypo{s = \Gamma(e)}
        \hypo{e \in \mathcal{A}}
        \infer2{\type{n}{e}{s}}
    \end{prooftree}
\end{equation*}

\subsubsection*{Operators}

\begin{itemize}[leftmargin=*]
    \setlength{\itemsep}{2em}

    \item $\oplus \in \binOp$:

          \begin{equation*}
              \begin{prooftree}
                  \hypo{\type{n}{a}{s_a}}
                  \hypo{\type{n}{b}{s_b}}
                  \hypo{s = \max\{s_a, s_b\}}
                  \infer3{\type{n}{a \oplus b}{s}}
              \end{prooftree}
          \end{equation*}

    \item $\oplus \in \unOp$:

          \begin{equation*}
              \begin{prooftree}
                  \hypo{\type{n}{e}{s}}
                  \infer1{\type{n}{\oplus e}{s}}
              \end{prooftree}
          \end{equation*}

    \item $\oplus \in \cast$:

          \begin{equation*}
              \begin{prooftree}
                  \hypo{\type{s}{e}{s}}
                  \infer1{\type{n}{\oplus(e)}{s}}
              \end{prooftree}
          \end{equation*}

    \item $\oplus \in \comp$:

          \begin{equation*}
              \begin{prooftree}
                  \hypo{\type{s}{a}{s_a}}
                  \hypo{\type{s}{b}{s_b}}
                  \hypo{s = \max\{s_a, s_b\}}
                  \infer3{\type{n}{a \oplus b}{1}}
              \end{prooftree}
          \end{equation*}

    \item $\oplus \in \logic$:

          \begin{equation*}
              \begin{prooftree}
                  \hypo{\type{s_a}{a}{s_a}}
                  \hypo{\type{s_b}{b}{s_b}}
                  \infer2{\type{n}{a \oplus b}{1}}
              \end{prooftree}
          \end{equation*}

    \item $\oplus \in \red$:

          \begin{equation*}
              \begin{prooftree}
                  \hypo{\type{s}{e}{s}}
                  \infer1{\type{n}{\oplus e}{1}}
              \end{prooftree}
          \end{equation*}

    \item $\oplus \in \shift$:

          \begin{equation*}
              \begin{prooftree}
                  \hypo{\type{n}{a}{s}}
                  \hypo{\type{s_b}{b}{s_b}}
                  \infer2{\type{n}{a \oplus b}{s}}
              \end{prooftree}
          \end{equation*}

    \item $\oplus \in \assignBinOp$:

          \begin{equation*}
              \begin{prooftree}
                  \hypo{s = \Phi(l)}
                  \hypo{\type{n_e}{e}{s_e}}
                  \hypo{n_e = \max\{s_a, s\}}
                  \infer3{\type{n}{l \oplus e}{s}}
              \end{prooftree}
          \end{equation*}

    \item $\oplus \in \assignShift$:

          \begin{equation*}
              \begin{prooftree}
                  \hypo{s = \Phi(l)}
                  \hypo{\type{s_e}{e}{s_e}}
                  \infer2{\type{n}{l \oplus e}{s}}
              \end{prooftree}
          \end{equation*}

    \item If expression:

          \begin{equation*}
              \begin{prooftree}
                  \hypo{\type{s_e}{e}{s_e}}
                  \hypo{\type{n}{a}{s_a}}
                  \hypo{\type{n}{b}{s_b}}
                  \hypo{s = \max\{s_a, s_b\}}
                  \infer4{\type{n}{e \texttt{?} a \texttt{:} b}{s}}
              \end{prooftree}
          \end{equation*}

    \item Concatenation:

          \begin{equation*}
              \begin{prooftree}
                  \hypo{\type{s_1}{e_1}{s_1}}
                  \hypo{\dots}
                  \hypo{\type{s_k}{e_k}{s_k}}
                  \hypo{s = \sum_{i=1}^{k} s_i}
                  \infer4{\type{n}{\texttt{\{}e_1, \texttt{\dots}, e_k \texttt{\}}}{s}}
              \end{prooftree}
          \end{equation*}

    \item Replication:

          \begin{equation*}
              \begin{prooftree}
                  \hypo{i \in \mathbb{N}}
                  \hypo{\type{s_e}{e}{s_e}}
                  \hypo{s = i \times s_e}
                  \infer3{\type{n}{\texttt{\{}i~e\texttt{\}}}{s}}
              \end{prooftree}
          \end{equation*}
\end{itemize}

\end{document}
