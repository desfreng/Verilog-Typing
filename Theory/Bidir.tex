\documentclass{article}
\usepackage[a4paper, margin=1cm]{geometry}
\usepackage[english]{babel}
\usepackage[mono=false]{libertine}

\usepackage{hyperref}
\usepackage{csquotes}
\usepackage{ascii}
\usepackage{enumitem}

\usepackage{amssymb}
\usepackage{amsmath}
\usepackage{amsfonts}
\usepackage{mathtools}

\usepackage{ebproof}

\allowdisplaybreaks{}

\newcommand{\mytilde}{\raisebox{-.7ex}{\asciitilde{}}}

\renewcommand{\epsilon}{\varepsilon}

\newcommand{\binOp}{\ensuremath{\left\{
      \texttt{+}, \texttt{-}, \texttt{*}, \texttt{/}, \texttt{\%}, \texttt{\&},
      \texttt{|}, \texttt{\^{}}, \texttt{\^{}}\mytilde, \mytilde\texttt{\^{}}
      \right\}}}
\newcommand{\unOp}{\ensuremath{\left\{
      \texttt{+}, \texttt{-}, \texttt{++}, \texttt{--}, \mytilde
    \right\}}}
\newcommand{\cast}{\ensuremath{\left\{\texttt{\$signed}, \texttt{\$unsigned}\right\}}}
\newcommand{\comp}{\ensuremath{\left\{
      \texttt{===}, \texttt{!==}, \texttt{==?}, \texttt{!=?}, \texttt{==},
      \texttt{!=}, \texttt{>}, \texttt{>=}, \texttt{<}, \texttt{<=}
    \right\}}}
\newcommand{\logic}{\ensuremath{\left\{\texttt{\&\&}, \texttt{||}, \texttt{->}, \texttt{<->}\right\}}}
\newcommand{\red}{\ensuremath{\left\{
      \texttt{\&}, \mytilde\texttt{\&}, \texttt{|}, \mytilde\texttt{|}, \texttt{\^{}},
      \mytilde\texttt{\^{}}, \texttt{\^{}}\mytilde, \texttt{!}
    \right\}}}
\newcommand{\shift}{\ensuremath{\left\{
      \texttt{>>}, \texttt{<}\texttt{<}, \texttt{**}, \texttt{>>>},
      \texttt{<}\texttt{<}\texttt{<}
    \right\}}}
\newcommand{\assignBinOp}{\ensuremath{\left\{
      \texttt{=}, \texttt{+=}, \texttt{-=}, \texttt{*=}, \texttt{/=},
      \texttt{\%=}, \texttt{\&=}, \texttt{|=}, \texttt{\^{}=}
    \right\}}}
\newcommand{\assignShift}{\ensuremath{\left\{
      \texttt{<}\texttt{<=}, \texttt{>>=},
      \texttt{<}\texttt{<}\texttt{<=}, \texttt{>>>=}
    \right\}}}

\pagestyle{empty}

\renewcommand{\S}{\ensuremath{\Rightarrow}}
\newcommand{\C}{\ensuremath{\Leftarrow}}
% \newcommand{\seq}[1]{\ensuremath{\Gamma, \Phi \vdash #1}}

\newcommand{\s}[2]{#1 \S{} #2}
\renewcommand{\c}[2]{#1 \C{} #2}

\begin{document}

\section*{Verilog Size Checking}

\subsection*{Expressions}

\begin{itemize}
    \item $\mathcal{A}$ is the set of atoms in an expression. In the AST, they correspond to leaves. In System-Verilog, they refer to what the standard calls an ``operand'' (variables, integers, function calls, slices of a variable, etc.).
    \item $\mathcal{R}$ is the set of resizable expressions. It corresponds to the expression whose top level is one of: atom, cast, comparisons, logic operation, reduction, assignments, shift assignments, concatenation, replication and inside operation.
    \item $\mathcal{E}$ is the set of System-Verilog expressions. We have $\mathcal{A} \subset \mathcal{E}$. An expression either contains an expression or is an atom.
\end{itemize}

\subsection*{Rules}
We use the following notations:
\begin{itemize}
    \item $\Gamma$ maps atoms to their size,
    \item $\Phi$ maps lvalues to their size,
    \item The statement $\s{e}{t}$ means ``$e$ has a size $t$'',
    \item The statement $\c{e}{t}$ means ``$e$ can be resized to $t$''.
\end{itemize}



\subsubsection*{Base case}

\begin{equation*}
    \begin{prooftree}
        \hypo{\Gamma(e) = s}
        \hypo{e \in \mathcal{A}}
        \infer2[Atom\S]{\s{e}{s}}
    \end{prooftree}
\end{equation*}


\subsubsection*{Resize case}

\begin{equation*}
    \begin{prooftree}
        \hypo{\s{e}{s}}
        \hypo{s \leqslant t}
        \hypo{e \in \mathcal{R}}
        \infer3[Resize\C]{\c{e}{t}}
    \end{prooftree}
\end{equation*}

\subsubsection*{Operators}

\begin{itemize}[leftmargin=*]
    \setlength{\itemsep}{2em}

    \item $\oplus \in \binOp$:

          \begin{align*}
               &
              \begin{prooftree}
                  \hypo{\s{a}{t}}
                  \hypo{\c{b}{t}}
                  \infer2[LBinOp\S]{\s{a\oplus b}{t}}
              \end{prooftree}
               &
               &
              \begin{prooftree}
                  \hypo{\c{a}{t}}
                  \hypo{\s{b}{t}}
                  \infer2[RBinOp\S]{\s{a\oplus b}{t}}
              \end{prooftree}
               &
               &
              \begin{prooftree}
                  \hypo{\c{a}{t}}
                  \hypo{\c{b}{t}}
                  \infer2[BinOp\C]{\c{a\oplus b}{t}}
              \end{prooftree}
          \end{align*}

    \item $\oplus \in \unOp$:

          \begin{align*}
               &
              \begin{prooftree}
                  \hypo{\s{e}{t}}
                  \infer1[UnOp\S]{\s{\oplus e}{t}}
              \end{prooftree}
               &
               &
              \begin{prooftree}
                  \hypo{\c{e}{t}}
                  \infer1[UnOp\C]{\c{\oplus e}{t}}
              \end{prooftree}
          \end{align*}

    \item $\oplus \in \cast$:

          \begin{equation*}
              \begin{prooftree}
                  \hypo{\s{e}{t}}
                  \infer1[Cast\S]{\s{\oplus(e)}{t}}
              \end{prooftree}
          \end{equation*}

    \item $\oplus \in \comp$:

          \begin{align*}
               &
              \begin{prooftree}
                  \hypo{\s{a}{t}}
                  \hypo{\c{b}{t}}
                  \infer2[LCmp\S]{\s{a \oplus b}{1}}
              \end{prooftree}
               &
               &
              \begin{prooftree}
                  \hypo{\c{a}{t}}
                  \hypo{\s{b}{t}}
                  \infer2[RCmp\S]{\s{a \oplus b}{1}}
              \end{prooftree}
          \end{align*}

    \item $\oplus \in \logic$:

          \begin{equation*}
              \begin{prooftree}
                  \hypo{\s{a}{t_a}}
                  \hypo{\s{b}{t_b}}
                  \infer2[Logic\S]{\s{a \oplus b}{1}}
              \end{prooftree}
          \end{equation*}

    \item $\oplus \in \red$:

          \begin{equation*}
              \begin{prooftree}
                  \hypo{\s{e}{t}}
                  \infer1[Red\S]{\s{\oplus e}{1}}
              \end{prooftree}
          \end{equation*}


    \item $\oplus \in \shift$:

          \begin{align*}
               &
              \begin{prooftree}
                  \hypo{\s{a}{t}}
                  \hypo{\s{b}{t_b}}
                  \infer2[Shift\S]{\s{a \oplus b}{t}}
              \end{prooftree}
               &
               &
              \begin{prooftree}
                  \hypo{\c{a}{t}}
                  \hypo{\s{b}{t_b}}
                  \infer2[Shift\C]{\c{a \oplus b}{t}}
              \end{prooftree}
          \end{align*}

    \item $\oplus \in \assignBinOp$:

          \begin{align*}
               &
              \begin{prooftree}
                  \hypo{\phi(l) = t}
                  \hypo{\c{e}{t}}
                  \infer2[LAssign\S]{\s{l \oplus e}{t}}
              \end{prooftree}
               &
               &
              \begin{prooftree}
                  \hypo{\phi(l) = t}
                  \hypo{\s{e}{t_e}}
                  \hypo{t < t_e}
                  \infer3[RAssign\S]{\s{l \oplus e}{t}}
              \end{prooftree}
          \end{align*}


    \item $\oplus \in \assignShift$:

          \begin{equation*}
              \begin{prooftree}
                  \hypo{\phi(l) = t}
                  \hypo{\s{e}{t_e}}
                  \infer2[AssignShift\S]{\s{l \oplus e}{t}}
              \end{prooftree}
          \end{equation*}


    \item If expression:

          \begin{align*}
               &
              \begin{prooftree}
                  \hypo{\s{e}{t_e}}
                  \hypo{\s{a}{t}}
                  \hypo{\c{b}{t}}
                  \infer3[LCond\S]{\s{e \texttt{?} a \texttt{:} b}{t}}
              \end{prooftree}
               &
               &
              \begin{prooftree}
                  \hypo{\s{e}{t_e}}
                  \hypo{\c{a}{t}}
                  \hypo{\s{b}{t}}
                  \infer3[RCond\S]{\s{e \texttt{?} a \texttt{:} b}{t}}
              \end{prooftree}
               &
               &
              \begin{prooftree}
                  \hypo{\s{e}{t_e}}
                  \hypo{\c{a}{t}}
                  \hypo{\c{b}{t}}
                  \infer3[Cond\C]{\c{e \texttt{?} a \texttt{:} b}{t}}
              \end{prooftree}
          \end{align*}


    \item Concatenation:

          \begin{equation*}
              \begin{prooftree}
                  \hypo{\s{e_1}{t_1}}
                  \hypo{\dots}
                  \hypo{\s{e_k}{t_k}}
                  \hypo{t = t_1 + \cdots + t_k}
                  \infer4[Concat\S]{\s{\texttt{\{}e_1, \texttt{\dots}, e_k \texttt{\}}}{t}}
              \end{prooftree}
          \end{equation*}


    \item Replication:

          \begin{equation*}
              \begin{prooftree}
                  \hypo{i \in \mathbb{N}}
                  \hypo{\s{e}{t_e}}
                  \hypo{t = i \times t_e}
                  \infer3[Repl\S]{\s{\texttt{\{}i~e\texttt{\}}}{t}}
              \end{prooftree}
          \end{equation*}


    \item Inside:

          \begin{gather*}
              \begin{prooftree}
                  \hypo{\s{a}{t}}
                  \hypo{\c{e_1}{t}}
                  \hypo{\dots}
                  \hypo{\c{e_k}{t}}
                  \infer4[LInside\S]{\s{a~\texttt{inside}~\texttt{\{}e_1, \texttt{\dots}, e_k \texttt{\}}}{1}}
              \end{prooftree}
              \\[1em]
              \begin{prooftree}
                  \hypo{i \in \{1, \dots, k\}}
                  \hypo{\c{a}{t}}
                  \hypo{\c{e_1}{t}}
                  \hypo{\dots}
                  \hypo{\c{e_{i-1}}{t}}
                  \hypo{\s{e_i}{t}}
                  \hypo{\c{e_{i+1}}{t}}
                  \hypo{\dots}
                  \hypo{\c{e_k}{t}}
                  \infer9[RInside\S]{\s{a~\texttt{inside}~\texttt{\{}e_1, \texttt{\dots}, e_k \texttt{\}}}{1}}
              \end{prooftree}
          \end{gather*}
\end{itemize}

\end{document}
