\documentclass{article}
\usepackage[a4paper, margin=1cm]{geometry}
\usepackage[english]{babel}
\usepackage[mono=false]{libertine}

\usepackage{hyperref}
\usepackage{csquotes}
\usepackage{ascii}
\usepackage{enumitem}

\usepackage{amssymb}
\usepackage{amsmath}
\usepackage{amsfonts}
\usepackage{mathtools}

\usepackage{ebproof}

\allowdisplaybreaks{}

\DeclareMathOperator{\Path}{Path}

\DeclareMathOperator{\Binary}{Binary}
\DeclareMathOperator{\Unary}{Unary}
\DeclareMathOperator{\Ternary}{Ternary}
\DeclareMathOperator{\Narry}{Narry}

\newcommand{\mytilde}{\raisebox{-.7ex}{\asciitilde{}}}

\renewcommand{\epsilon}{\varepsilon}

\newcommand{\binOp}{\ensuremath{\left\{
      \texttt{+}, \texttt{-}, \texttt{*}, \texttt{/}, \texttt{\%}, \texttt{\&},
      \texttt{|}, \texttt{\^{}}, \texttt{\^{}}\mytilde, \mytilde\texttt{\^{}}
      \right\}}}
\newcommand{\unOp}{\ensuremath{\left\{
      \texttt{+}, \texttt{-}, \texttt{++}, \texttt{--}, \mytilde
    \right\}}}
\newcommand{\cast}{\ensuremath{\left\{\texttt{\$signed}, \texttt{\$unsigned}\right\}}}
\newcommand{\comp}{\ensuremath{\left\{
      \texttt{===}, \texttt{!==}, \texttt{==?}, \texttt{!=?}, \texttt{==},
      \texttt{!=}, \texttt{>}, \texttt{>=}, \texttt{<}, \texttt{<=}
    \right\}}}
\newcommand{\logic}{\ensuremath{\left\{\texttt{\&\&}, \texttt{||}, \texttt{->}, \texttt{<->}\right\}}}
\newcommand{\red}{\ensuremath{\left\{
      \texttt{\&}, \mytilde\texttt{\&}, \texttt{|}, \mytilde\texttt{|}, \texttt{\^{}},
      \mytilde\texttt{\^{}}, \texttt{\^{}}\mytilde, \texttt{!}
    \right\}}}
\newcommand{\shift}{\ensuremath{\left\{
      \texttt{>>}, \texttt{<}\texttt{<}, \texttt{**}, \texttt{>>>},
      \texttt{<}\texttt{<}\texttt{<}
    \right\}}}

\pagestyle{empty}

\renewcommand{\S}{\ensuremath{\Rightarrow}}
\newcommand{\C}{\ensuremath{\Leftarrow}}

\newcommand{\s}[3]{#1 \S{} #2 \dashv{} #3}
\renewcommand{\c}[3]{#1 \C{} #2 \dashv{} #3}

\begin{document}

\section*{Verilog Size Checking}

\subsection*{Path}
Definition of paths and their relation to sub-expressions.

\subsection*{Expressions}

\begin{itemize}
    \item $\mathcal{A}$ is the set of atoms in an expression. In the AST, they correspond to leaves. In System-Verilog, they refer to what the standard calls an ``operand'' (variables, integers, function calls, slices of a variable, etc.).
    \item $\mathcal{R}$ is the set of resizable expressions. It corresponds to the expression whose top level is one of: atom, cast, comparisons, logic operation, reduction, assignments, shift assignments, concatenation, replication and inside operation.
    \item $\mathcal{E}$ is the set of System-Verilog expressions. We have $\mathcal{A} \subset \mathcal{E}$. An expression either contains an expression or is an atom.
\end{itemize}

\subsection*{Rules}
We use the following notations:
\begin{itemize}
    \item $\Gamma$ maps atoms to their size,
    \item $\Phi$ maps lvalues to their size,
    \item The statement $\s{e}{t}{f}$ means ``$e$ has size $t$, with $f$ mapping each sub-expressions of $e$ to their size'',
    \item The statement $\c{e}{t}{f}$ means ``$e$ can be resized to $t$, with $f$ mapping each sub-expressions of $e$ to their size''.
\end{itemize}


\subsubsection*{Function combinator}
\begin{equation*}
    \renewcommand\arraystretch{1.1}
    \begin{array}{ccc}
        \Binary(t, f, g) : \left\{
        \begin{array}{ccc}
            {[]}               & \mapsto & t     \\
            0 \dblcolon p  & \mapsto & f (p) \\
            1 \dblcolon p & \mapsto & g (p)
        \end{array}
        \right.
         & \hspace{1em} &
        \Unary(f, t) : \left\{
        \begin{array}{ccc}
            {[]}             & \mapsto & t     \\
            0 \dblcolon p & \mapsto & f (p)
        \end{array}
        \right.
        \\
        \Narry(t, f_1, \dots, f_k) : \left\{
        \begin{array}{ccc}
            {[]}                 & \mapsto & t       \\
            i \dblcolon p & \mapsto & f_i (p)
        \end{array}
        \right.
         & \hspace{1em} &
        \Ternary(t, f, g, h) : \left\{
        \begin{array}{ccc}
            {[]}               & \mapsto & t     \\
            0 \dblcolon p   & \mapsto & f (p) \\
            1 \dblcolon p  & \mapsto & g (p) \\
            2] \dblcolon p & \mapsto & h (p)
        \end{array}
        \right.
    \end{array}
\end{equation*}

\subsubsection*{Base case}

\begin{equation*}
    \begin{prooftree}
        \hypo{\Gamma(e) = s}
        \hypo{e \in \mathcal{A}}
        \infer2[Atom\S]{\s{e}{s}{\{[]\mapsto s\}}}
    \end{prooftree}
\end{equation*}


\subsubsection*{Resize case}

\begin{equation*}
    \begin{prooftree}
        \hypo{\s{e}{s}{f}}
        \hypo{s \leqslant t}
        \hypo{e \in \mathcal{R}}
        \infer3[Resize\C]{\c{e}{t}{f\big[{[]}\mapsto t\big]}}
    \end{prooftree}
\end{equation*}

\subsubsection*{Operators}

\begin{itemize}[leftmargin=*]
    \setlength{\itemsep}{2em}

    \item $\oplus \in \binOp$:

          \begin{align*}
               &
              \begin{prooftree}
                  \hypo{\s{a}{t}{f_a}}
                  \hypo{\c{b}{t}{f_b}}
                  \infer2[LBinOp\S]{\s{a\oplus b}{t}{\Binary(t, f_a, f_b)}}
              \end{prooftree}
               &
               &
              \begin{prooftree}
                  \hypo{\c{a}{t}{f_a}}
                  \hypo{\s{b}{t}{f_b}}
                  \infer2[RBinOp\S]{\s{a\oplus b}{t}{\Binary(t, f_a, f_b)}}
              \end{prooftree}
          \end{align*}
          \vspace*{.5em}
          \begin{equation*}
              \begin{prooftree}
                  \hypo{\c{a}{t}{f_a}}
                  \hypo{\c{b}{t}{f_b}}
                  \infer2[BinOp\C]{\c{a\oplus b}{t}{\Binary(t, f_a, f_b)}}
              \end{prooftree}
          \end{equation*}

    \item $\oplus \in \unOp$:

          \begin{align*}
               &
              \begin{prooftree}
                  \hypo{\s{e}{t}{f}}
                  \infer1[UnOp\S]{\s{\oplus e}{t}{\Unary(t, f)}}
              \end{prooftree}
               &
               &
              \begin{prooftree}
                  \hypo{\c{e}{t}{f}}
                  \infer1[UnOp\C]{\c{\oplus e}{t}{\Unary(t, f)}}
              \end{prooftree}
          \end{align*}

    \item $\oplus \in \cast$:

          \begin{equation*}
              \begin{prooftree}
                  \hypo{\s{e}{t}{f}}
                  \infer1[Cast\S]{\s{\oplus(e)}{t}{\Unary(t, f)}}
              \end{prooftree}
          \end{equation*}

    \item $\oplus \in \comp$:

          \begin{align*}
               &
              \begin{prooftree}
                  \hypo{\s{a}{t}{f_a}}
                  \hypo{\c{b}{t}{f_b}}
                  \infer2[LCmp\S]{\s{a \oplus b}{1}{\Binary(1, f_a, f_b)}}
              \end{prooftree}
               &
               &
              \begin{prooftree}
                  \hypo{\c{a}{t}{f_a}}
                  \hypo{\s{b}{t}{f_b}}
                  \infer2[RCmp\S]{\s{a \oplus b}{1}{\Binary(1, f_a, f_b)}}
              \end{prooftree}
          \end{align*}

    \item $\oplus \in \logic$:

          \begin{equation*}
              \begin{prooftree}
                  \hypo{\s{a}{t_a}{f_a}}
                  \hypo{\s{b}{t_b}{f_b}}
                  \infer2[Logic\S]{\s{a \oplus b}{1}{\Binary(1, f_a, f_b)}}
              \end{prooftree}
          \end{equation*}

    \item $\oplus \in \red$:

          \begin{equation*}
              \begin{prooftree}
                  \hypo{\s{e}{t}{f}}
                  \infer1[Red\S]{\s{\oplus e}{1}{\Unary(1, f)}}
              \end{prooftree}
          \end{equation*}


    \item $\oplus \in \shift$:

          \begin{align*}
               &
              \begin{prooftree}
                  \hypo{\s{a}{t}{f_a}}
                  \hypo{\s{b}{t_b}{f_b}}
                  \infer2[Shift\S]{\s{a \oplus b}{t}{\Binary(t, f_a, f_b)}}
              \end{prooftree}
               &
               &
              \begin{prooftree}
                  \hypo{\c{a}{t}{f_a}}
                  \hypo{\s{b}{t_b}{f_b}}
                  \infer2[Shift\C]{\c{a \oplus b}{t}{\Binary(t, f_a, f_b)}}
              \end{prooftree}
          \end{align*}

    \item Assigment:

          \begin{align*}
               &
              \begin{prooftree}
                  \hypo{\phi(l) = t}
                  \hypo{\c{e}{t}{f}}
                  \infer2[LAssign\S]{\s{(l~\texttt{=}~e)}{t}{\Unary(t, f)}}
              \end{prooftree}
               &
               &
              \begin{prooftree}
                  \hypo{\phi(l) = t}
                  \hypo{\s{e}{t_e}{f}}
                  \hypo{t < t_e}
                  \infer3[RAssign\S]{\s{(l~\texttt{=}~e)}{t}{\Unary(t, f)}}
              \end{prooftree}
          \end{align*}


    \item If expression:

          \begin{align*}
               &
              \begin{prooftree}
                  \hypo{\s{e}{t_e}{f_e}}
                  \hypo{\s{a}{t}{f_a}}
                  \hypo{\c{b}{t}{f_b}}
                  \infer3[LCond\S]{\s{e \texttt{?} a \texttt{:} b}{t}{\Ternary(t, f_e, f_a, f_b)}}
              \end{prooftree}
               &
               &
              \begin{prooftree}
                  \hypo{\s{e}{t_e}{f_e}}
                  \hypo{\c{a}{t}{f_a}}
                  \hypo{\s{b}{t}{f_b}}
                  \infer3[RCond\S]{\s{e \texttt{?} a \texttt{:} b}{t}{\Ternary(t, f_e, f_a, f_b)}}
              \end{prooftree}
          \end{align*}
          \vspace*{.5em}
          \begin{equation*}
              \begin{prooftree}
                  \hypo{\s{e}{t_e}{f_e}}
                  \hypo{\c{a}{t}{f_a}}
                  \hypo{\c{b}{t}{f_b}}
                  \infer3[Cond\C]{\c{e \texttt{?} a \texttt{:} b}{t}{\Ternary(t, f_e, f_a, f_b)}}
              \end{prooftree}
          \end{equation*}


    \item Concatenation:

          \begin{equation*}
              \begin{prooftree}
                  \hypo{\s{e_1}{t_1}{f_i}}
                  \hypo{\dots}
                  \hypo{\s{e_k}{t_k}{f_k}}
                  \hypo{t = t_1 + \cdots + t_k}
                  \infer4[Concat\S]{\s{\texttt{\{}e_1, \texttt{\dots}, e_k \texttt{\}}}{t}
                      {\Narry(t, f_1, \dots, f_k)}}
              \end{prooftree}
          \end{equation*}


    \item Replication:

          \begin{equation*}
              \begin{prooftree}
                  \hypo{i \in \mathbb{N}}
                  \hypo{\s{e}{t_e}{f}}
                  \hypo{t = i \times t_e}
                  \infer3[Repl\S]{\s{\texttt{\{}i~e\texttt{\}}}{t}{\Unary(t, f)}}
              \end{prooftree}
          \end{equation*}
\end{itemize}

\end{document}
